%%%% RESUMO
%%
%% Apresentação concisa dos pontos relevantes de um texto, fornecendo uma visão rápida e clara do conteúdo e das conclusões do
%% trabalho.

\begin{resumoutfpr}%% Ambiente resumoutfpr
    Tendo em vista a ascenção dos sistemas embarcados no decorrer dos últimos 10 anos, este trabalho vem com o objetivo de desenvolver um sistema embarcado que seja capaz de reconhecer um gás através de suas propriedades físico-quimica de forma acústica, emitindo uma faixa de frequência dentro de um tubo, dessa forma o gás dentro do tubo entra em ressoância, a ressonância causa amplitudes mais altas, essas amplitudes são detectadas em um microfone acoplado ao final do tubo . Esse sinal é amostrado por um sistema embarcado que executa um algoritmo de transformada rápida de Fourier para o reconhecimento do gás através das suas propriedades físico-quimica. Este experimento é comumente chamado de experimento de kundt, onde tem-se um meio dentro de um tubo e esse tubo é excitado com ondas longitudinais.
\end{resumoutfpr}